
\thispagestyle{empty}
\vspace{-1cm}\begin{figure}
	\centering
	\includegraphics[width=1\linewidth]{C:/Users/user/itmo/inf/lab6/1_str_top}
\end{figure}
\begin{minipage}{0.25\linewidth}\vspace{-0.1cm}
	Главный редактор\newline\textit{академик} И. К. Кикоин
	\newline\newline
	Первый заместитель\newline главного редактора\newline \textit{академик} А.Н. Колмогоров\newline\newline
	
	\textbf{Редакционная коллегия:}\vspace{0.05cm}
	
	М. И. Башмаков.
	
	С. Т. Беляев.
	
	В. Г. Болтянский,
	
	Н. Б. Васильев,
	
	Ю. Н. Ефремов.
	
	В. Г. Зубов,
	
	П. Л. Капица,
	
	В. А. Кириллин.\vspace{0.1cm}
	
	\textit{главный художник}
	
	А. И. Климанов.
	
	С. М. Козел.\vspace{0.1cm}
	
	\textit{зам. главного редактора}
	
	В. А. Лешковцев.
	
	Л. Г. Макар-Лиманов
	
	А. И. Маркушевич.
	
	Н. А. Патрикеева,
	
	И. С. Петраков,
	
	Н. Х. Розов,
	
	А. П. Савин.
	
	И. Ш. Слободецкий.\vspace{0.1cm}
	
	\textit{зам. главного редактора}
	
	М Л. Смолянский.
	
	Я. А. Смородинский
	
	В. А. Фабрикант,
	
	А. Т. Цветков.
	
	М. П. Шаскольская,
	
	С. И. Шварцбурд,
	
	А. И. Ширшов.\vspace{0.5cm}
	
	\textbf{Редакция:}
	
	В. Н. Березин.
	
	А. Н. Виленкин,
	
	И. Н. Клумова.\vspace{0.1cm}
	
	\textit{художественный редактор}
	
	Т. М. Макарова,
	
	Н. А. Минц,
	
	Т. С. Петрова,
	
	В. А. Тихомирова.\vspace{0.1cm}
	
	\textit{зав. редакцией}
	
	Л. В. Чернова
\end{minipage}
\hfill
\vline
\hfill
\begin{minipage}{0.695\linewidth}\vspace{1cm}
	\begin{tikzpicture}[overlay, remember picture]
		\node[anchor=north east, yshift=-96mm, xshift=-100mm, font=\large\bfseries, align=right] at (current page.north east) {В НОМЕРЕ:};
		\draw[black, line width=0.3mm] ([xshift=-1cm, yshift=-101mm]current page.north east) -- ++(-\textwidth,0);
	\end{tikzpicture}
	\begin{tabular}{r p{11cm}}
		2 & Я. А. Смородинский. Сила Кориолиса \\
		9 & И. Н. Бронштейн. Парабола \\
		17 & В. Н. Ланге. Зачем топят печи? \\
		& \textbf{Лаборатория «Кванта»}\\
		19 &  М. И. Емельянов, А. М. Жарков, В. М Загайнов, \\
		& В. С. Маточкин. Суточное вращение Земли \\
		& \textbf{Математический кружок}\\
		21&А. П. Винниченко. Квадратичный треугольник и \\
		& непрерывные цепочки \\
		& \textbf{Задачник «Квант»}\\
		25 & Задачи М316---М320; Ф328---Ф332 \\
		27 & Решения задач М280---М285; Ф291---Ф295 \\
		&\textbf{Практикум абитуриента}\\
		37&С. Т. Берколайко. Использование неравенства Коши при\\
		&решении задач\\
		41&Г. Я. Мякишев. Электростатическое поле\\
		48&Г. В. Меледин, А. И. Ширшов. Новосибирский\\
		&государственный университет\\
		&\textbf{Рецензии, библиография}\\
		52&Д. Бородин. Школьникам о современной физике\\
		&\textbf{«Квант» для младших школьников}\\
		53&Задачи\\
		54&Е. Я. Гик. Математические игры на шахматной доске\\
		&\textbf{Ответы, указания, решения}\\
		60&\textbf{Смесь} (с. 18, 24, 47)\\
		\cline{2-2}
		&\\
		&\textit{\fontsize{7}{2}\selectfont На первой странице обложки вы видите известный фонтан «Дружба народов»(Москва, ВДНХ). Струи бьющей воды имеют весьма характерную форму - форму параболы. Об этой кривой и некоторых ее свойствах вы можете прочесть в статье «Парабола» (с. 9). Обратите внимание на то, что по сравнению с этой фотографией параболы в статье изображены «вниз головой», но пусть это вас не смущает: интересна сама кривая, и не ее расположение..}\vspace{-0.7cm}\\
		&\begin{flushright}
			\textit{\fontsize{6}{5}\selectfont Фото Д. 11. Германа.}
		\end{flushright}\vspace{-0.7cm}\\
		\cline{2-2}\vspace{-0.4cm}
		&\\
		& \textbf{\textit{\fontsize{7}{2}\selectfont \copyright Главная редакция физико-математической литературы}}\\
		&\vspace{-0.4cm}\hspace{0.36cm}\textbf{\textit{\fontsize{7}{2}\selectfont издательства «Наука», «Квант», 1975 год}}\\
	\end{tabular}
	\hfill
	\vline
	\hfill
\end{minipage}
