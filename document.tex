
\begin{figure}[H]
	\centering
	\fbox{\includegraphics[width=1\linewidth]{img_1}}
\end{figure}
\vspace{-0.2cm}
\begin{paracol}{2} % Создание двух колонок
	\setlength{\columnsep}{2em} % Расстояние между колонками
	\begin{leftcolumn}
		\hspace{-0.5cm}Рис 1. \\
	\end{leftcolumn}
	
	\begin{rightcolumn}	
		\hspace{-0.5cm}Рис 2.\\
	\end{rightcolumn}
	
\end{paracol}
\vspace{-0.6cm}

\begin{multicols}{2}\fontsize{14}{14}\selectfont
	З а д а ч а 5 (НГУ, 1968). \textit{\fontsize{14}{16}Расстояние между двумя параллельными прямыми равно \(h\). Третья прямая, параллельная данным, находится вне полосы между ними на расстоянии \(H\) от дальней. Отрезок \(AB\) перпендикулярен к прямым, а концы его лежат на первых двух прямых.}
	
	\textit{\fontsize{14}{16}Найти на третьей прямой точку \(M\) так, чтобы \(\varangle AMB\) был наибольшим.}
	
	Положение точки М на третьей прямой определяется, например, расстоянием от нее до точки \(K\) пересечения прямой \(AB\) с третьей прямой (рис. 2). Обозначим \(KM\) через \(x\). Для решения задачи достаточно найти то значение \(x\), для которого со- ответствующий угол \(\alpha\) = \(\varangle AMB\) достигает наибольшей величины.
	
	Введем вспомогательный угол \(\beta\) = \(\varangle KMA\) и выразим с помощью него \(\tg{a}\) в виде функции от \(x\). Используя формулу тангенса разности двух аргументов и определение тангенса острого угла, получим \(\tg{a} = \tg[(\alpha + \beta) - \beta] =\)\begin{center}\fontsize{15}{16}
		\fontsize{15}{16}=\(\frac{\tg{(\alpha + \beta)} - \tg{\beta}}{1 + \tg{(\alpha + \beta)}\cdot\tg{\beta}}\)\fontsize{15}{16}=\\
		=\fontsize{19}{16}\(\frac{\frac{H}{x}-\frac{H - h}{x}}{1 + \frac{H(H -h)}{x^2}}\)\fontsize{15}{16}=\fontsize{19}{16}\(\frac{h}{x + \frac{H(H - h)}{x}}\).
	\end{center}
	\fontsize{14}{14}\selectfont Отсюда видно, что \(\tg{a}\)(а, следовательно, и \(\alpha\)) достигает наибольшего значения, когда сумма \(x + \frac{H(H-h)}{x}\)\columnbreak\newline достигает наименьшего значения. Поскольку \(x\frac{H(H-h)}{x}\) постоянно, воспользуемся неравенством (2):\\
	\(x + \frac{H(H-h)}{x}\geqslant2\sqrt{x\frac{H(H-h)}{x}}\)=
	=\(2\sqrt{H(H-h)}\).\\
	Искомое значение \(x\) является положительным корнем уравнения \(x=\frac{H(H-h)}{x}\), то есть
	\vspace{-0.2cm}\begin{center}
		\(x = \sqrt{H(H-h)}\)
	\end{center}
	
	Искомую точку теперь можно найти с помощью несложного построения: опишем на отрезке \(KB\) как на диаметре полуокружность, которая пересечет вторую прямую (проходящую через \(A\)) в точке \(C\) (рис. 2), и отложим на третьей прямой отрезок \(KM_1\) = \(KC\); точка \(M_1\) --- искомая (докажите это самостоятельно).
	
	Нетрудно заметить, что точка \(M_2\), симметричная точке \(M_1\) относительно точки к, обладает тем же свойством, что и точка \(M_1\). Итак, искомых точек --- две: \(M_1\) и \(M_2\).
	
	З а д а ч а 6 (МГУ, 1971). \textit{Автомобиль едет от пункта \(A\) до пункта \(B\) с постоянной скоростью} 42\textit{ км/ч. В пункте \(B\) он переходит на равнозамедленное движение, причем за каждый час его скорость уменьшается на а км/ч, и едет так до полной остановки. Затем он сразу же начинает двигаться равноускоренно с ускорением \(\alpha\) км/ч².}
\end{multicols}%