
\begin{multicols}{2}\small
	3. \(3\sqrt[3]{2}/4\). Можно найти это значение r как максимальное при котором уравнения \(y = x^4, x^2 + (y-r)^2=r^2\) имеют общее решение, отличное от x=y=0. А можно кроме этих двух уравнений получить третье используя тот факт для критического значения окружность имеет с графиком \(y = x^4\) общие касательные(в некоторой точке, отличной от x=y=0, см.рис.13).\\\\
	Избранные задачи Московской физической олимпиады\\
	\vspace{-0.3mm}\rule{\linewidth}{0.3mm}
	\rule{\linewidth}{0.3mm}\vspace{-0.5cm}
	Первый тур\\
	9 класс\\
	1. \(v = \sqrt{a/M}t.\) 2. Горки разъезжаются в противоположные стороны с почти одинаковыми скоростями \(v = \sqrt{mgH/M}\).//
	3. \(T^2_1/T^2_2 = h_1/h_2.\)\\
	\textit{10 класс}\\
	1. Брусок притягивается к стенке с силой \(F = (2a^2l)/(pga^a)\)\\
	\columnbreak\newline
	\vspace{-0.3mm}\rule{\linewidth}{0.3mm}
	\rule{\linewidth}{0.3mm}\vspace{-0.5cm}
	Второй тур\\
	\textit{9 класс}\\
	1. См. таблицу, в которой \(t_0 = 1 с\)\\
	2. \(w_\mathrm{min} < w < w_\mathrm{max}\), где\\
	\vspace{-0.2cm}\fontsize{6}{8}\begin{align*}
		w_\mathrm{min} &= \sqrt{\frac{g\left(\sqrt{h\left(2R - h\right)}-\mu\left(R-h\right)\right)}{\sqrt{h\left(2R-h\right)}\left(\left(R-h\right)+\mu\sqrt{h\left(2R-h\right)}\right)}} \quad \text{при } \mu<\frac{\sqrt{h\left(2R-h\right)}}{R-h}\\
		w_\mathrm{min} &= 0 \quad \text{при } \mu \geq\sqrt{h\left(2R-h\right)}/\left(R-h\right)\\
		w_\mathrm{min} &= \sqrt{\frac{g\left(\sqrt{h\left(2R-h\right)}+\mu\left(R-h\right)\right)}{\sqrt{h\left(2R-h\right)}\left(\left(R-h\right)-\mu\sqrt{h\left(2R-h\right)}\right)}}\\
		w_\mathrm{min} &= \infty \quad \text{при } \mu\geq(R-h)/\sqrt{h(2R-h)}
	\end{align*}
\end{multicols}
\begin{flushright}
	Таблица
\end{flushright}\vspace{-0.3cm}
\begin{tabular}{|m{3.7cm}|>{\centering\arraybackslash}p{4.2cm}|>{\centering\arraybackslash}p{4.2cm}|>{\centering\arraybackslash}p{4.2cm}|}
	\hline
	Возможный случай & При каких \(s\) возможен & Начальная скорость & Путь, пройденный за вторую секунду \\
	\hline
	В течении двух секунд камень движется вверх &\(s>\frac{3}{2}gt^2_0\) &\(\frac{s}{t_0}+\frac{gt_0}{2}\)&\(s-gt^2_0\)\\
	\hline
	Камень поворачивает в течении второй секунды & \(\frac{gt^2_0}{2}<s<\frac{3}{2}gt^2_0\)&\(\frac{s}{t_0}+\frac{gt_0}{2}\)&\(\frac{5}{4}gt^2_0-2s+\frac{s^2}{gt^2_0}\)\\
	\hline
	Камень поворачивает в течении первой секунды&\(\frac{gt^2_0}{4}<s<\frac{gt^2_0}{2}\)&\(\frac{gt_0+\sqrt{4gs-g^2t^2_0}}{2}\)&\(\frac{2gt^2_0-\sqrt{4gst^2_0-g^2t^4_0}}{2}\)\\
	\hline
	Камень поворачивает в течении первой секунды&\(\frac{gt^2_0}{4}<s<\frac{gt^2_0}{2}\)&\(\frac{gt_0-\sqrt{4gs-g^2t^2_0}}{2}\)&\(\frac{2gt^2_0-\sqrt{4gst^2_0-g^2_0t^4_0}}{2}\)\\
	\hline
\end{tabular}
