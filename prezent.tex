\documentclass{beamer}
\usepackage[utf8]{inputenc}
\usepackage[english,russian]{babel}
\usepackage{graphicx}
\usepackage{tikz}
\usepackage{geometry}

\geometry{paperwidth=453.4px,paperheight=255px}

\setbeamercolor{footline}{fg=gray}
\setbeamertemplate{footline}{%
	\ifnum\value{framenumber}>1
	\makebox[\paperwidth][r]{\insertframenumber\hspace{7ex}}%
	\fi
	\vspace{2ex}%
}

\title{\\\vspace{2cm} Лекция 6.\\\vspace{0.1cm}\textcolor{black}{\textbf{Анализ предметной области и требования к ПО}\\ \textbf{Качество ПО и методы его контроля} \\ \textbf{Экстремальное программирование}}}
\author{\textcolor{blue}{Лектор}\newline\newline Елена Болдырева \hspace{2cm}eaboldyreva@itmo.ru}
\date{}

\begin{document}
	\begin{frame}
		\titlepage
		\begin{tikzpicture}[overlay, remember picture]
			\node[anchor=north east, yshift=-0.3cm, xshift=-0.5cm, font=\Large\bfseries, align=right] at (current page.north east) {\textcolor{blue}{И}\textcolor{gray}{нформатика}};
			\draw[gray, line width=0.2mm] ([yshift=-10mm]current page.north east) -- ([yshift=-10mm]current page.north west);
			
		\end{tikzpicture}
	\end{frame}
	
	\begin{frame}{\vspace{0.2cm}\newline\textbf{ПЛАН:}}
		\begin{tikzpicture}[overlay, remember picture]
			\draw[gray, line width=0.2mm] ([xshift=3cm, yshift=-10mm]current page.north east) -- ++(-\textwidth,0);
		\end{tikzpicture}
		\tableofcontents
		\vspace{-1.5cm}\textbf{Раздел 4.} Анализ предметной области и требования к ПО\\
		\hspace{0.8cm}Анализ предметной области\\
		\hspace{0.8cm}Выделение и анализ требований\\
		\hspace{0.8cm}Варианты использования\\
		\textbf{Раздел 5.} Качество ПО и методы его контроля\\
		\hspace{0.8cm}Качество ПО\\
		\hspace{0.8cm}Метода контроля качества\\
		\hspace{0.8cm}Ошибки в программах\\
		\textbf{Раздел 6.} Архитектура ПО\\
		\hspace{0.8cm}Анализ области решений\\
		\hspace{0.8cm}Разработка архитектуры\\
		\hspace{0.8cm}UML-диаграммы\\
	\end{frame}
	
	\begin{frame}
		\begin{tikzpicture}[overlay, remember picture]
			\draw[gray, line width=0.2mm] ([xshift=3cm, yshift=-10mm]current page.north east) -- ++(-\textwidth,0);
		\end{tikzpicture}
		\begin{center}
			\textbf{\textcolor{blue}{РАЗДЕЛ 1. АНАЛИЗ ПРЕДМЕТНОЙ ОБЛАСТИ\\И ТРЕБОВАНИЯ К ПО}}
		\end{center}
	\end{frame}
	
	\begin{frame}{}
		\begin{tikzpicture}[overlay, remember picture]
			\node[anchor=north east, yshift=-2.5mm, xshift=-0.5cm, font=\Large\bfseries, align=right] at (current page.north east) {\textcolor{gray}{Бизнес-моделирование}};
			\draw[gray, line width=0.2mm] ([xshift=3cm, yshift=-10mm]current page.north east) -- ++(-\textwidth,0);
		\end{tikzpicture}
		
		\vspace{0.4cm}
		
		\begin{center}
			\textbf{Область ответственности будущей системы. Схема Захмана}
		\end{center}
		
		\vspace{-2.5mm} % Измените это значение по вашему усмотрению
		
		\begin{figure}
			\centering
			\includegraphics[width=0.7\linewidth]{image007}
			\label{fig:image007}
		\end{figure}
	\end{frame}
	
	\begin{frame}{}
		\begin{tikzpicture}[overlay, remember picture]
			\node[anchor=north east, yshift=-0.3cm, xshift=-0.5cm, font=\Large\bfseries, align=right] at (current page.north east) {\textcolor{blue}{ДИАГРАММЫ ПОТОКОВ ДАННЫХ}};
			\node[anchor=north east, yshift=-1cm, xshift=-0.5cm, font=\Large\bfseries, align=right] at (current page.north east) {\textcolor{blue}{(DATA FLOW DIAGRAMS)}};
			\draw[gray, line width=0.2mm] ([xshift=3cm, yshift=-9mm]current page.north east) -- ++(-\textwidth,0);
		\end{tikzpicture}
		\vspace{1cm}
		\begin{flushleft}
			\begin{minipage}{0.4\textwidth}
				Схема деятельности компании в\\
				нотации Йордана-ДеМарко.\newline
				
				Схема деятельности компании в\\
				нотации Гэйна-Сарсона.\newline
				
				Диаграмма сущностей и связей.
			\end{minipage}%
			\begin{minipage}{0.6\textwidth}
				\includegraphics[width=1\linewidth]{4_3}
			\end{minipage}
		\end{flushleft}
		
	\end{frame}
	
	
	
	
\end{document}
